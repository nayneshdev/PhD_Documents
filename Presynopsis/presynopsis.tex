\documentclass[draft,11pt]{scrreprt}
\usepackage[numbib,nottoc]{tocbibind}
\usepackage{csquotes}
\usepackage{setspace}
\onehalfspacing
\usepackage{booktabs}
\usepackage{graphicx}
\begin{document}
\pagestyle{empty}
\begin{center}
\large\textbf{{SARDAR VALLABHBHAI NATIONAL INSTITUTE OF TECHNOLOGY, SURAT}}\\
\vfill
\includegraphics[scale=0.25]{svnit}\\ \vfill
\textbf{DEPARTMENT OF APPLIED PHYSICS}\\ \vfill
\textbf{Pre Synopsis Seminar Report}\\  \vfill
\normalsize
of the Ph. D. thesis entitled\\ \vfill

\large
\textbf{Study of Heavy-Light and Heavy-Heavy Flavoured Mesons}\\
\vfill
\normalsize
Proposed to be submitted\\
in partial fulfilment of the Degree of\\
\vfill
\textbf{DOCTOR OF PHILOSOPHY\\ in \\ PHYSICS}\\
\vfill By\\

\begin{tabular}{lcl}
\textbf{Student Name} & : & {\textbf{NAYNESHKUMAR DEVLANI}} \\
& & \textbf{Roll No. D09AS302} \\
\textbf{Supervisor} & : & \textbf{DR. AJAY KUMAR RAI}\\
\textbf{Month and Year} & : & \textbf{September 2012}



\end{tabular}
\end{center}

\newpage
\begin{minipage}{0.3\textwidth}
\begin{center}
\includegraphics[scale=0.25]{svnit}
\end{center}

\end{minipage}
\hfill
\begin{minipage}{0.65\textwidth}
\large Department of Applied Physics\\
Sardar Vallabhbhai National Institute of Technology,\\
Surat-395 007, Gujarat, India.

\end{minipage}
\vspace*{10pt}

\rule{\columnwidth}{1.5pt}
\vspace*{10pt}
\begin{center}
\large\textbf{\underline{CERTIFICATE}}
\end{center} 
\vspace*{10pt}
\hfill \textbf{Date: \today}\\[15pt]


 This is to certify that the pre-synopsis entitled \textbf{Spectroscopy and Decay Properties of Heavy-Light and Heavy-Heavy Flavoured Mesons: A Comparative Study}  is submitted by Mr.
NAYNESHKUMAR B. DEVLANI in partial fulfilment for the award of degree of Doctor of Philosophy in Physics at S.V. National Institute of Technology, Surat is a record of his own work carried out under my supervision and guidance.

\vspace*{25pt}
\begin{flushright}
\begin{tabular}{lll}
\textbf{Supervisor} & \textbf{:} & \textbf{Dr. Ajay Kumar Rao}\\
& & \textbf{Asst. Prof.} APD\\
& & \textbf{SVNIT, Surat}

\end{tabular}
\end{flushright}

\newpage
\begin{minipage}{0.3\textwidth}
\begin{center}
\includegraphics[scale=0.25]{svnit}
\end{center}

\end{minipage}
\hfill
\begin{minipage}{0.65\textwidth}
\large Department of Applied Physics\\
Sardar Vallabhbhai National Institute of Technology,\\
Surat-395 007, Gujarat, India.

\end{minipage}
\vspace*{10pt}\\

\hrulefill



\begin{center}
\large\textbf{\underline{CERTIFICATE}}
\end{center} 
\vspace*{10pt}
\hfill \textbf{Date: \today}\\[15pt]


 This is to certify that the pre-synopsis entitled \textbf{Spectroscopy and Decay Properties of Heavy-Light and Heavy-Heavy Flavoured Mesons: A Comparative Study}  is submitted by Mr.
NAYNESHKUMAR B. DEVLANI in partial fulfilment for the award of degree of Doctor of Philosophy in Physics at S.V. National Institute of Technology, Surat is a record of his own work carried out under my supervision and guidance.

\vspace*{25pt}
\begin{flushright}
\begin{tabular}{lll}
\textbf{Chairman} & \textbf{:} & \textbf{Dr. }\\
& & \textbf{} \\
& & \textbf{SVNIT, Surat}\\
\addlinespace
\addlinespace
\addlinespace
\addlinespace
\addlinespace
\textbf{Supervisor} & \textbf{:} & \textbf{Dr. Ajay Kumar Rao}\\
& & \textbf{Asst. Prof.} APD\\
& & \textbf{SVNIT, Surat}\\
\addlinespace
\addlinespace
\addlinespace
\addlinespace
\addlinespace
\textbf{Examiner} & \textbf{:} & \textbf{Dr. Lalit Kumar Saini}\\
& & \textbf{Asst. Prof.} APD\\
& & \textbf{SVNIT, Surat}\\
\addlinespace
\addlinespace
\addlinespace
\addlinespace
\addlinespace
\textbf{Examiner} & \textbf{:} & \textbf{Dr. Vipul A Kheraj}\\
& & \textbf{Asst. Prof.} APD\\
& & \textbf{SVNIT, Surat}




\end{tabular}
\end{flushright}

\pagestyle{empty} %get rid of header/footer for toc page
\tableofcontents %put toc in
\cleardoublepage %start new page
\pagestyle{plain} % put headers/footers back on
\setcounter{page}{1} %reset the page counter

\part{Background of the study}
It is a well established fact that quarkonia which are bound states of heavy quark and anti-quark can be studied using methods of non-relativistic quantum mechanics with the help of QCD motivated potential models. While there also exist approaches in QCD to study the light quark composites but mesons containing one heavy quark and another light quark are interesting hadrons. They are like hydrogen atoms of QCD. The presence of light quark requires a relativistic treatment for these mesons. $L=1.0$ states of the open charmed and open beauty mesons are particularly interesting as none of the theoretical models have satisfactorily been able to reproduce their mass spectrum. It is suggested that the confinement interaction plays an important role in these mesons.

There are several QCD inspired potential models which are used successfully in the study of mass spectra of quarkonia. In particular a potential with a short range Coulmobic form and a linear long range confining interaction has proved to be of great generality for various mesons. Since the heavy-light mesons are like hydrogen atom it is interesting to compare the results obtained with the help of a variational approach which makes use of both Gaussian and Hydrogen like wavefunctions, since the former is well suited for linear potential form while the latter is more appropriate for Coulombic form. Thus within the variational approach results for the $D$, $D_s$, $B$, $B_s$ meson spectra are obtained using both the Hydrogen-like and Gaussian wavefunction, and the results are compared.

Recently, the BaBar collaboration, in inclusive $e^{+}e^{-}$ collisions, observed four new charmed states $D(2550)$, $D(2600)$, $D(2750)$, and $D(2760)$\cite{AmoSanchez2010a}. The isospin partners of the $D(2600)^{0}$ and $D(2760)^{0}$ were
also observed. BaBar collaboration in 2009, in inclusive $e^{+}e^{-}$ collisions also observed two new charmed-strange states $D_{s1}(2710)$ and $D_{sJ}(2860)$\cite{Aubert2009}. They also found an evidence of $D_{sJ}(3040)$. In case of open beauty mesons apart from well established states two $L=1$ candidate states were observed in the nineties. The $B_{J}^{*}(5732)$\cite{Abreu1995} and the $B_{sJ}^{\prime}(5850)$\cite{Akers1995}. The possible interpretation of these states and determination of their $J^{P}$ values is very important to single out a particular approach within quark models. 

The presence of light quark certainly makes the non-relativistic potential model based approach questionable, but it is interesting to study how far one can stretch the approach by incorporating the kinematic relativistic corrections within the Hamiltonian of these systems. In this work the kinematic relativistic corrections are included by expanding the kinetic energy term of the Hamiltonian to justly treat the light degree of freedom. Particle antiparticle mixing phenomenon has played a pivotal role in weak interaction physics. This work also determines the mixing parameters from predictions obtained using potential model approach. The overall objective of the thesis can be put as:
\begin{enumerate}
\item Comparative study of properties of heavy-light and heavy-heavy mesons using variational approach with Hydrogen-like and Gaussian wavefunctions.
\item Extending the potential model approach to heavy-light mesons by including the kinematic relativistic correction to the Hamiltonia of these systems.
\end{enumerate}
\part{Structure of the Thesis}


The thesis entitled \textbf{Spectroscopy and Decay Properties of Heavy-Light and Heavy-Heavy Flavoured Mesons: A Comparative Study} consists of five chapters.
\chapter{Introduction}
This chapter provides an intoduction to the subject of particle physics with a particular emphasis on hadron physics. Different aspects of standard model are briefly discussed.
This chapter provides an overview to the current status of research in the field of hadron physics both from an experimental and theoretical aspects. Recent experimental measurements by various facilities worldwide are discussed as well as ongoing studies in the relevant areas are outlined. Different approaches used in QCD for the determination of meson properties are outlined. Particular emphasis is placed on Potential Models. A general overview of potential models is given. The open charmed and open beauty mesons are discussed. Recent experimental detections of new resonances of these mesons and underlying physics is discussed. This chapter also provides motivation for the thesis work. The problem with $L=1$ states of these mesons as well as the possible interpretation of newly observed states of these mesons is discussed. 

\chapter{Theoretical Overview\label{chap:ch1}}
This chapter gives basic theoretical framework behind the potential models. Mesons containing heavy quarks with a mass $> \Lambda_{QCD}$ can be treated non-relativistically. So a systemetic approach based on the SM and th Schrodinger equation is valid employing a quark antiquark potential model. As per QCD the theory of the strong interaction of the standard model quarks must permanently remain confined into bound states. The potential must employ two basic characteristics the so called confinement and the asysmptotic freedom. According to these at short distances the quarks move relatively freely but at larger distance they must attract strongly to remain permanenltl The potential employed within the present work is Coulomb plus power law term. Various other potential model schemes are discussed in detail providing necessary base for the present study. 

\chapter{Mass spectra of heavy-light flavored and heavy-heavy flavored mesons}
Relativistic hamiltonian for the system is employed. Gaussian as well as Hydrogen-like radial wave functions are listed. Variational approach used in present work is thoroughly discussed. The procedure for fixing the parameters for the study is clearly outlined in this chapter. This chapter provides detailed calculations for the mass spectra of the heavy-light as well as heavy-heavy mesons using the method outlined in Chapter \ref{chap:ch1}. Kinematic relativistic corrections are employed within the hamiltonian. Spin-orbit, spin-spin and tensor interactions are used to obtain $L=0,1,2$ masses of these mesons. These calculated results are compared with well-known other theoretical model predictions as well as recent experimental measurements. The newly observed states are also compared with various theoretical predictions.

\chapter{E1 and M1 transitions of heavy-light and heavy-heavy mesons}
In this chapter widths of the E1 dipole transitions between S and P wave masses and M1 dipole transitions between various S wave states are evaluated within non-relativistic scheme by making use of the initial and final state wavefunctions. The calculations are done for the heavy-light as well as heavy-heavy mesons. Results are compared with other theoretical model predictions and experimental results. 

\chapter{Decay properties}
Decay constants which are important parameters for many weak decays are evaluated using the Van-Royen Weisskopf formula both for the pseudoscalar and vector, heavy-light as well as heavy-heavy mesons. QCD corrections are emplyed to these decay constants. Results obtained are compared with recent experimental measurements as well as theoretical predictions. Particle-antiparticle mixing parameters which are of fundamental importance in weak interaction physics, are evaluated using the spectroscopic parameters of the present work for the case of $B$ and $B_s$  mesons. The results are in excellent agreement with recent experimental measurements.
\chapter{Summary, conclusion and future work}

The present work studied the properties of the heavy-light and heavy-heavy flavored mesons using the potential model approach. The potential model formulation which is valid mainly for non-relativistic mesons is extended by incorporating kinematic relativistic corrections. Mass spectra, decay properties, e1 and M1 decay widths as well as mixing parameters are evaluated for various. mesons.

\chapter*{Major References}
\bibliographystyle{plain}
\bibliography{myref.bib}
\newpage

\chapter*{Details of Credit and Research Progress Seminar}
\cite{devlani2012spectroscopy}
\cite{devlani2011p}
\cite{devlani2009spectra}
\cite{devlani2010spectrum}
\cite{rai2009decay}

\chapter*{Research papers published/accepted/communicated in National/International journals}
\begin{itemize}
\item International Journals
\begin{enumerate}
\item Nayneshkumar Devlani and Ajay Kumar Rai, \textit{Spectroscopy and decay properties of the $D_s$ meson}, Phys. Rev. D. \textbf{84}, pp. 074030, Oct 2011.
\item N. Devlani and A.K. Rai, \textit{Spectrscopy and decay properties of $B$ and $B_s$ mesons}, Eur, Phys. J. A \textbf{48(7)}, pp. 1-12, 2012.
\end{enumerate}
\item National Journals

\begin{enumerate}
\item Nayneshkumar Devlani and Ajay Kumar Rai, \textit{S-Wave Masses and Decay Properties of $D$ and $D_s$ Mesons} in PRAJNA,
Journal of Pure and Applied Sciences, Vol. 18, 145 - 147, (2010). (ISSN 0975 -
2595).

\end{enumerate}
\end{itemize}

\chapter*{Research papers presented in National/International Conferences}

\begin{itemize}
\item International Conference
\begin{enumerate}
\item ``Spectra and Decay Constants of $D$ and $D_s$ Mesons'', Proceedings of the International Symp. on Nucl. Phys. Vol. 54, pp 506 (2009).

\item ``Decay Rates of Excited States of Charmonium'', Proceedings of the International Symp. on Nucl. Phys. vol. 54, pp 506 (2009).
\end{enumerate}
\item National Conference
\begin{enumerate}
\item ``Spectrum and Decay Properties of $B$ and $B_s$ Mesons'', Proceedings of the DAE, Symp. on Nucl. Phys. Vol. 55, pp 543 (2010).



\item \enquote{$P-$wave masses of the $D_s$ Meson}, Proceedings of the DAE Symp. on Nucl. Phys. 56, pp 878 (2011).
\end{enumerate}
\end{itemize}
\bibliographystyle{plain}
\bibliography{myref.bib}
\end{document}


